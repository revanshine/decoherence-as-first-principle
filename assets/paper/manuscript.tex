% Options for packages loaded elsewhere
\PassOptionsToPackage{unicode}{hyperref}
\PassOptionsToPackage{hyphens}{url}
\documentclass[
]{article}
\usepackage{xcolor}
\usepackage{amsmath,amssymb}
\setcounter{secnumdepth}{-\maxdimen} % remove section numbering
\usepackage{iftex}
\ifPDFTeX
  \usepackage[T1]{fontenc}
  \usepackage[utf8]{inputenc}
  \usepackage{textcomp} % provide euro and other symbols
\else % if luatex or xetex
  \usepackage{unicode-math} % this also loads fontspec
  \defaultfontfeatures{Scale=MatchLowercase}
  \defaultfontfeatures[\rmfamily]{Ligatures=TeX,Scale=1}
\fi
\usepackage{lmodern}
\ifPDFTeX\else
  % xetex/luatex font selection
\fi
% Use upquote if available, for straight quotes in verbatim environments
\IfFileExists{upquote.sty}{\usepackage{upquote}}{}
\IfFileExists{microtype.sty}{% use microtype if available
  \usepackage[]{microtype}
  \UseMicrotypeSet[protrusion]{basicmath} % disable protrusion for tt fonts
}{}
\makeatletter
\@ifundefined{KOMAClassName}{% if non-KOMA class
  \IfFileExists{parskip.sty}{%
    \usepackage{parskip}
  }{% else
    \setlength{\parindent}{0pt}
    \setlength{\parskip}{6pt plus 2pt minus 1pt}}
}{% if KOMA class
  \KOMAoptions{parskip=half}}
\makeatother
\usepackage{longtable,booktabs,array}
\usepackage{calc} % for calculating minipage widths
% Correct order of tables after \paragraph or \subparagraph
\usepackage{etoolbox}
\makeatletter
\patchcmd\longtable{\par}{\if@noskipsec\mbox{}\fi\par}{}{}
\makeatother
% Allow footnotes in longtable head/foot
\IfFileExists{footnotehyper.sty}{\usepackage{footnotehyper}}{\usepackage{footnote}}
\makesavenoteenv{longtable}
% definitions for citeproc citations
\NewDocumentCommand\citeproctext{}{}
\NewDocumentCommand\citeproc{mm}{%
  \begingroup\def\citeproctext{#2}\cite{#1}\endgroup}
\makeatletter
 % allow citations to break across lines
 \let\@cite@ofmt\@firstofone
 % avoid brackets around text for \cite:
 \def\@biblabel#1{}
 \def\@cite#1#2{{#1\if@tempswa , #2\fi}}
\makeatother
\newlength{\cslhangindent}
\setlength{\cslhangindent}{1.5em}
\newlength{\csllabelwidth}
\setlength{\csllabelwidth}{3em}
\newenvironment{CSLReferences}[2] % #1 hanging-indent, #2 entry-spacing
 {\begin{list}{}{%
  \setlength{\itemindent}{0pt}
  \setlength{\leftmargin}{0pt}
  \setlength{\parsep}{0pt}
  % turn on hanging indent if param 1 is 1
  \ifodd #1
   \setlength{\leftmargin}{\cslhangindent}
   \setlength{\itemindent}{-1\cslhangindent}
  \fi
  % set entry spacing
  \setlength{\itemsep}{#2\baselineskip}}}
 {\end{list}}
\usepackage{calc}
\newcommand{\CSLBlock}[1]{\hfill\break\parbox[t]{\linewidth}{\strut\ignorespaces#1\strut}}
\newcommand{\CSLLeftMargin}[1]{\parbox[t]{\csllabelwidth}{\strut#1\strut}}
\newcommand{\CSLRightInline}[1]{\parbox[t]{\linewidth - \csllabelwidth}{\strut#1\strut}}
\newcommand{\CSLIndent}[1]{\hspace{\cslhangindent}#1}
\setlength{\emergencystretch}{3em} % prevent overfull lines
\providecommand{\tightlist}{%
  \setlength{\itemsep}{0pt}\setlength{\parskip}{0pt}}
\usepackage{bookmark}
\IfFileExists{xurl.sty}{\usepackage{xurl}}{} % add URL line breaks if available
\urlstyle{same}
\hypersetup{
  pdftitle={Decoherence as First Principle: A Framework for Emergent Forces, Dark Matter, and Cosmological Structure},
  pdfauthor={R. Evanshine},
  hidelinks,
  pdfcreator={LaTeX via pandoc}}

\title{Decoherence as First Principle: A Framework for Emergent Forces,
Dark Matter, and Cosmological Structure}
\author{R. Evanshine}
\date{2025}

\begin{document}
\maketitle

\section{Abstract}\label{abstract}

We propose a novel interpretation of cosmological evolution grounded in
the concept of decoherence as the primary physical mechanism underlying
the emergence of classical reality (\citeproc{ref-zurek2009}{Zurek,
2009}). In this view, the Big Bang is reinterpreted not as an explosion
from a singularity, but as a Boltzmann-like fluctuation
(\citeproc{ref-boltzmann1896}{Boltzmann, 1896};
\citeproc{ref-carroll2008}{Carroll, 2010}) that initiated a sustainable
decoherence cascade---what we term a \emph{decoherence bootstrap}. This
bootstrap sets the conditions under which forces, particles, and
spacetime structures emerge as stable pointer states of a quantum system
undergoing localized decoherence (\citeproc{ref-gellmann1990}{Gell-Mann
and Hartle, 1990}).

We argue that gravity represents the earliest and most fundamental
decoherence channel (\citeproc{ref-diosi1987}{Diósi, 1987};
\citeproc{ref-penrose1996}{Penrose, 1996}), while other
forces---particularly electromagnetism---emerge only after sufficient
decoherence chain length has stabilized their respective interaction
bases. In this model, dark matter is naturally reinterpreted as
mass-energy that decohered gravitationally prior to the stabilization of
the electromagnetic pointer basis. This provides an explanatory
framework that accounts for dark matter's gravitational effects without
requiring interaction with the electromagnetic sector
(\citeproc{ref-bertone2018}{Bertone and Hooper, 2018}).

We conclude by exploring the implications of this framework for black
holes (\citeproc{ref-danielson2023}{Danielson, Satishchandran and Wald,
2023}), quantum aliasing, and conservation principles in cosmology.
\newpage

\section{1. Introduction}\label{introduction}

The quantum--classical boundary has long been treated as an epistemic
artifact---a transition between microscopic indeterminacy and
macroscopic determinism brought about by environmental measurement. In
this work, we take a stronger stance: \textbf{decoherence is
ontological}. The collapse of quantum superpositions is not merely a
convenient description for observers but is instead a fundamental
dynamical feature of reality's fabric, setting the cadence at which
``real'' events occur.

This perspective gains urgency when considered alongside two of modern
physics' most persistent puzzles: the unification of quantum mechanics
with general relativity, and the nature of dark matter. Both domains are
deeply concerned with the \emph{structure of reality} at different
scales, yet their current theoretical languages remain largely
incompatible. We propose that treating decoherence as an ontological
process---occurring in multiple ``pointer bases'' (e.g., gravitational,
electromagnetic)---offers a route to reconciling these descriptions.

We introduce the \textbf{Decoherence Bootstrap Hypothesis}, which posits
that the early universe emerged from a singularity-like state through a
self-sustaining decoherence process that progressively lengthened the
``collapse length'' in different interaction bases. This
sequence---gravitational → strong → electroweak →
electromagnetic---provides a natural ordering for the emergence of
forces without requiring ad hoc inflationary dynamics, while still
accommodating the observed imprints in the cosmic microwave background.

From this perspective, black holes represent the inverse limit of the
bootstrap: as matter spirals toward a singularity, the collapse length
in the gravitational basis diverges, terminating the sequence of
decoherence events and returning localized regions of reality to
``singularity space.''

In what follows, we:

\begin{itemize}
\tightlist
\item
  Develop a quantitative framework linking collapse length to both
  decoherence rate and gravitational time dilation.
\item
  Reinterpret dark matter as \textbf{mass decohered in the gravitational
  basis but not in the electromagnetic basis}, explaining its purely
  gravitational influence.
\item
  Show how quantum phenomena can be recast as \emph{aliasing artifacts}
  arising from the finite sampling frequency of reality itself.
\item
  Connect the emergence of forces to conservation principles, proposing
  that the ordering of their appearance reflects an optimal pathway for
  dissipating the universe's initial energy density.
\end{itemize}

By situating decoherence as the engine of cosmic structure, we aim to
offer both a conceptual unification of gravity and quantum mechanics and
a falsifiable set of predictions that touch observational cosmology,
quantum information theory, and high-energy physics.
(\citeproc{ref-zwicky1933}{Zwicky, 1933}; \citeproc{ref-rubin1970}{Rubin
and Ford Jr, 1970}; \citeproc{ref-zurek2003}{Zurek, 2003};
\citeproc{ref-clowe2006}{Clowe \emph{et al.}, 2006};
\citeproc{ref-schlosshauer2007}{Schlosshauer, 2007};
\citeproc{ref-carroll2008}{Carroll, 2010})

\section{2. The Decoherence Bootstrap
Hypothesis}\label{the-decoherence-bootstrap-hypothesis}

We propose that the Big Bang can be understood as a Boltzmann-like
fluctuation (\citeproc{ref-boltzmann1896}{Boltzmann, 1896},
\citeproc{ref-boltzmann1896english}{1966}) that achieved a
\textbf{sustainable decoherence cascade} --- a self-propagating sequence
of events in which quantum superpositions were progressively reduced to
stable classical pointer states (\citeproc{ref-zurek2003}{Zurek, 2003};
\citeproc{ref-schlosshauer2007}{Schlosshauer, 2007}). This process,
which we term the \emph{decoherence bootstrap}, reframes the origin of
the universe not as an uncontrolled explosion from a singularity, but as
the initiation of a chain reaction in the ``Collapse Length'' of
reality.

In the earliest instants after this fluctuation, the \textbf{decoherence
chain length} was vanishingly short: superpositions could not persist
over any meaningful scale, and no stable physical structures or forces
could emerge. This brevity imposed a unification of all interactions ---
the system lacked the persistence to differentiate between bases such as
electromagnetism, the strong and weak nuclear forces, and gravitation
(\citeproc{ref-weinberg1974}{Weinberg, 1974};
\citeproc{ref-guth1981}{Guth, 1981}). In this regime, the only
physically meaningful structure was the quantum state of the whole.

The bootstrap succeeded because the Collapse Length began to grow ---
whether through an intrinsic thermodynamic instability
(\citeproc{ref-prigogine1984}{Prigogine and Stengers, 1984}),
amplification of quantum fluctuations
(\citeproc{ref-mukhanov1981}{Mukhanov and Chibisov, 1981}), or a deep
statistical bias in the space of possible microstates. As decoherence
events chained together over longer separations, specific interaction
bases could stabilize. Gravity, being universally coupled and
insensitive to charge or spin (\citeproc{ref-misner1973}{Misner, Thorne
and Wheeler, 1973}), is hypothesized to have been the first decoherence
channel to achieve persistence. Subsequent forces emerged only once the
decoherence chain could support their more delicate pointer states
(\citeproc{ref-zurek2003}{Zurek, 2003}).

The sustainability of this bootstrap is nontrivial. Not all fluctuations
would have resulted in a growing Collapse Length; most would recede into
singularity space, never establishing a classical history. The fact that
our universe's decoherence chain lengthened, diversified, and stabilized
interaction channels implies an underlying asymmetry --- possibly rooted
in the low-entropy boundary conditions at the origin
(\citeproc{ref-penrose1989}{Penrose, 1989}). This marks the Big Bang as
the \textbf{boundary condition where Collapse Length was at its
shortest}, in symmetry with black hole singularities as the
\textbf{boundary condition where Collapse Length becomes infinite}.

This perspective suggests that the ``laws of physics'' are not immutable
axioms but \textbf{emergent features} of a decoherence network that
reached a self-sustaining phase. The forces, particles, and spacetime
fabric we observe are therefore contingent on the success of this
bootstrap. In the absence of sustained decoherence, reality collapses
back into the undifferentiated potential of singularity space, as is
hypothesized to occur in black hole interiors
(\citeproc{ref-hawking1976}{Hawking, 1976}).

\section{3. Gravity as the Primary Decoherence
Basis}\label{gravity-as-the-primary-decoherence-basis}

In the decoherence bootstrap framework, gravity occupies a unique
position as the \textbf{first and most fundamental decoherence channel}.
This primacy arises from three key properties:

\begin{enumerate}
\def\labelenumi{\arabic{enumi}.}
\item
  \textbf{Universality of coupling} --- Unlike the other fundamental
  interactions, gravity couples to \emph{all} forms of mass--energy
  without exception or shielding. No matter the quantum state's
  composition, it necessarily interacts gravitationally with the rest of
  the universe.
\item
  \textbf{Geometric embodiment} --- In general relativity, gravity is
  not a force in the gauge-field sense but the manifestation of
  spacetime curvature itself. Any emergent classical structure must
  therefore inherit a gravitational \emph{pointer basis} before other
  interaction channels can stabilize.
\item
  \textbf{Dominance at Planck-scale densities} --- In the earliest
  universe, energy densities approached the Planck scale, where
  gravitational effects become overwhelming relative to other forces. At
  these scales, spacetime curvature fluctuations dominate the
  decoherence environment (\citeproc{ref-penrose1994}{Penrose, 1994}).
\end{enumerate}

\begin{center}\rule{0.5\linewidth}{0.5pt}\end{center}

\subsection{3.1 Gravity as the First Classical
Frame}\label{gravity-as-the-first-classical-frame}

The curvature of spacetime is arguably the \textbf{first classical
observable} to emerge. This observable defines the topology and metric
properties of the universe, providing a \emph{stage} upon which all
subsequent interactions play out. In this sense, the gravitational
pointer basis is not merely a passive backdrop---it is the
\textbf{structural skeleton} of classical reality.

In the standard view, early-universe density perturbations are generated
and amplified during inflation (\citeproc{ref-guth1981}{Guth, 1981};
\citeproc{ref-linde1982}{Linde, 1982};
\citeproc{ref-baumann2009}{Baumann, 2009}), then \emph{locked in} via
gravitational decoherence before the electromagnetic pointer basis
existed. In the decoherence bootstrap framework, this ``locking in'' can
be explained even without inflation: \textbf{at the earliest moments the
Collapse Length in the gravitational basis is extremely short}, i.e.,
decoherence events occur at ultra-high frequency across the primordial,
nonlocal pre-classical state. This rapid, quasi-simultaneous
gravitational decoherence can establish large-scale correlations without
a separate inflationary phase, while still yielding the observed
uniformity and coherence of the early universe
(\citeproc{ref-steinhardt2002}{Steinhardt and Turok, 2002};
\citeproc{ref-brandenberger2011}{Brandenberger, 2011}).

Under either picture, the seeds of cosmic structure---visible in the
cosmic microwave background (CMB) anisotropies---reflect a
gravitationally-set architecture established before electromagnetic
observables existed.

\emph{Boundary conditions summary:} the Big Bang corresponds to the
\textbf{short-Collapse Length} limit (high-frequency decoherence
initiating the chain), while black holes correspond to the
\textbf{Collapse Length \(\to \infty\)} limit (termination of the chain
in singularity space).

\begin{center}\rule{0.5\linewidth}{0.5pt}\end{center}

\subsection{3.2 Pre-Electromagnetic Decoherence
Epochs}\label{pre-electromagnetic-decoherence-epochs}

Before the electromagnetic basis stabilized, mass--energy could decohere
purely in the gravitational channel. Such matter would remain
\textbf{invisible} to any later EM-based detection, interacting only
through its gravitational influence. This provides a natural,
first-principles path to the dark matter reinterpretation developed in
Section 5 (\citeproc{ref-zwicky1933}{Zwicky, 1933};
\citeproc{ref-rubin1970}{Rubin and Ford Jr, 1970};
\citeproc{ref-rubin1980}{Vera C. Rubin and Jr., 1980}).

We define the \textbf{gravitational pointer basis} as the set of states
whose decoherence is determined solely by spacetime curvature,
independent of gauge interactions. In this framework, the earliest
decohered mass--energy becomes ``dark'' not because it is exotic, but
because it is \textbf{pre-EM classical}---its history is locked in
before electromagnetic observables existed.

\begin{center}\rule{0.5\linewidth}{0.5pt}\end{center}

In summary, gravity's universality, geometric nature, and dominance in
the high-energy early universe position it as the first and most stable
decoherence channel. The gravitational pointer basis shapes the earliest
classical structure and predetermines the large-scale architecture of
the universe before other forces even emerge.

\section{4. Emergence of the Standard Model
Forces}\label{emergence-of-the-standard-model-forces}

In the \emph{decoherence bootstrap} framework, the emergence of the
Standard Model forces can be understood as the sequential stabilization
of distinct pointer bases as the Collapse Length of reality increased
from its near-zero value at the Big Bang.

At the earliest times, when the Collapse Length was vanishingly short,
no force could meaningfully ``separate'' from the undifferentiated
interaction network. All interactions were unified because quantum
states could not persist over the spatial or temporal scales necessary
to define distinct coupling channels
(\citeproc{ref-weinberg1974}{Weinberg, 1974};
\citeproc{ref-guth1981}{Guth, 1981}). In this regime, the universe
behaved as a single, maximally entangled system with no meaningful
subdivision into forces or particles.

As the Collapse Length grew --- whether due to thermodynamic
irreversibility (\citeproc{ref-prigogine1984}{Prigogine and Stengers,
1984}), amplification of vacuum fluctuations
(\citeproc{ref-mukhanov1981}{Mukhanov and Chibisov, 1981}), or
statistical bias in the space of possible configurations --- the first
interaction to stabilize was \textbf{gravitation}. Gravity's
universality (coupling to all forms of energy-momentum) and
insensitivity to internal quantum numbers such as charge or spin
(\citeproc{ref-misner1973}{Misner, Thorne and Wheeler, 1973}) meant it
could maintain coherence over the shortest scales. This gravitational
pointer basis effectively ``froze in'' the large-scale geometry of
spacetime, setting the stage for further differentiation.

The \textbf{strong nuclear force} is hypothesized to have stabilized
next. Confinement at low energies and asymptotic freedom at high
energies (\citeproc{ref-gross1973}{Gross and Wilczek, 1973};
\citeproc{ref-politzer1973}{Politzer, 1973}) imply that color charge
could maintain a robust pointer basis once the Collapse Length exceeded
the confinement scale, allowing hadronic matter to form without
immediate reversion to a fully entangled state.

The \textbf{electroweak force} emerged as the Collapse Length further
expanded. Initially unified at high energies
(\citeproc{ref-glashow1961}{Glashow, 1961};
\citeproc{ref-weinberg1967}{Weinberg, 1967}), the electroweak
interaction underwent spontaneous symmetry breaking via the Higgs
mechanism (\citeproc{ref-higgs1964}{Higgs, 1964}) once decoherence
events could persist over scales comparable to the Higgs field
correlation length. This bifurcation produced the photon as the stable
gauge boson of electromagnetism --- an interaction whose pointer basis
is the most fragile and thus the last to stabilize in the early
universe.

In this view, the ``force unification epochs'' of conventional cosmology
are reinterpreted as \textbf{decoherence-limited regimes}, each defined
by the maximum Collapse Length achievable at that epoch. Rather than
treating unification solely as a function of temperature or energy
scale, the decoherence bootstrap posits that force differentiation is
constrained by the rate at which reality can sustain independent pointer
bases.

This ordering --- gravity → strong → electroweak → electromagnetic ---
mirrors the increasing fragility of each interaction's pointer basis. It
also provides a natural explanation for why gravitation remains so
resistant to unification with the other forces: its decoherence
threshold lies at the very foundation of the bootstrap process, anchored
in the shortest Collapse Length regime the universe has ever
experienced.

\section{5. A Natural Interpretation of Dark
Matter}\label{a-natural-interpretation-of-dark-matter}

We propose that the phenomenon currently labeled \emph{dark matter} can
be understood not as an exotic, undiscovered particle species, but as
\textbf{mass-energy that decohered in the gravitational pointer basis
prior to the stabilization of the electromagnetic pointer basis}. In
other words, it is matter that is ``classical'' with respect to gravity,
but remains effectively quantum---unmeasured and non-interacting---in
the electromagnetic sector.

This interpretation naturally explains the observed properties of dark
matter:

\begin{itemize}
\tightlist
\item
  \textbf{Gravitational Interaction} --- Such mass would still curve
  spacetime and participate fully in gravitational clustering, producing
  the galaxy rotation curves first inferred by Zwicky in galaxy clusters
  (\citeproc{ref-zwicky1933}{Zwicky, 1933}) and refined in his later
  work (\citeproc{ref-zwicky1937}{Zwicky, 1937}), confirmed on galactic
  scales by Rubin and Ford (\citeproc{ref-rubin1970}{Rubin and Ford Jr,
  1970}; \citeproc{ref-rubin1980}{Vera C. Rubin and Jr., 1980}), and
  most strikingly demonstrated in the Bullet Cluster, where
  gravitational lensing maps reveal a clear separation of mass from
  luminous matter (\citeproc{ref-clowe2006}{Clowe \emph{et al.}, 2006}).
\item
  \textbf{Electromagnetic Invisibility} --- Without decoherence in the
  EM basis, these structures neither emit, absorb, nor scatter photons,
  accounting for their non-detection in optical, radio, or X-ray
  surveys.
\item
  \textbf{Persistence Across Cosmic Time} --- Once decohered in gravity
  but not in EM, this matter remains ``dark'' indefinitely, since
  decoherence in one basis does not force decoherence in another unless
  a cross-basis interaction occurs.
\end{itemize}

\subsubsection{Predictions and Testable
Implications}\label{predictions-and-testable-implications}

\begin{enumerate}
\def\labelenumi{\arabic{enumi}.}
\tightlist
\item
  \textbf{Epoch Signatures} --- If dark matter represents a
  pre-EM-decoherence population, its large-scale distribution should
  preserve a subtle imprint of the gravitational potential landscape
  that existed before electromagnetic stabilization. This could manifest
  as anisotropies or preferred clustering modes in the cosmic web.
\item
  \textbf{Mass Distribution Asymmetry} --- Regions of high gravitational
  decoherence density in the early universe would seed both dark matter
  halos and the baryonic structures we observe, but with differing
  mass-to-light ratios that can be mapped statistically
  (\citeproc{ref-blumenthal1984}{Blumenthal \emph{et al.}, 1984}).
\item
  \textbf{No Direct Detection in EM-Based Experiments} --- Any detection
  of dark matter through purely electromagnetic coupling would falsify
  this interpretation, since such interaction would imply EM
  decoherence.
\end{enumerate}

By recasting dark matter as a \emph{basis-dependent decoherence
artifact}, we eliminate the need for speculative WIMPs, axions, or other
beyond--Standard Model particles, while preserving all observed
gravitational phenomena. This reframing also integrates seamlessly into
the decoherence bootstrap model, placing the origin of dark matter in a
specific cosmological stage---after gravitational decoherence but before
electromagnetic decoherence.

The Bullet Cluster, in this context, becomes a striking snapshot of the
universe's ``two-step'' decoherence history. The collision shows the
gravitationally decohered mass components---dark matter halos---passing
through each other with minimal interaction, while the
electromagnetically decohered baryonic matter is shock-heated and
slowed. This clean separation is exactly what we would expect if
gravitational decoherence occurred first in the early universe,
establishing large-scale mass distributions, with electromagnetic
decoherence---and thus photon-coupled interactions---emerging later. In
the decoherence bootstrap framework, the Bullet Cluster serves not just
as evidence for non-luminous mass, but as a fossil record of the
temporal ordering of decoherence events that shaped the cosmos.

\section{6. Black Holes and the Decoherence
Limit}\label{black-holes-and-the-decoherence-limit}

Black holes represent the terminal state of decoherence within our
observable universe (\citeproc{ref-penrose1965}{Penrose, 1965};
\citeproc{ref-hawking1973}{Hawking and Ellis, 1973}). As gravitational
collapse proceeds beyond the event horizon, matter spirals inward
through increasingly extreme regimes of classical spacetime until the
separation between decoherence events---the \textbf{Collapse
Length}---stretches toward infinity (\citeproc{ref-zurek2003}{Zurek,
2003}). In this limit, no further decoherence occurs; the chain of
classical reality frames terminates, and all that remains is
\emph{singularity space}.

This framing makes it possible to distinguish the black hole singularity
from the Big Bang in precise terms. Both mark a transition between
singularity space and classical reality, but in opposite directions: the
Big Bang is the \textbf{boundary condition where Collapse Length was at
its shortest}, launching reality; the black hole singularity is the
\textbf{boundary condition where Collapse Length becomes infinite},
ending it. The Big Bang corresponds to the initiation of a sustainable
decoherence chain---a Boltzmann-like fluctuation
(\citeproc{ref-boltzmann1896}{Boltzmann, 1896},
\citeproc{ref-boltzmann1896english}{1966}) that successfully
bootstrapped a self-propagating sequence of classical frames. The black
hole singularity marks the \emph{end} of such a chain, where decoherence
ceases because the Collapse Length has become infinite.

Energy density alone does not differentiate these transitions; in the
mathematical limit, both approach the same curvature singularity
(\citeproc{ref-misner1973}{Misner, Thorne and Wheeler, 1973}). The
decisive difference lies in entropy. The Big Bang began in a
low-entropy, high-potential configuration capable of sustaining
branching decoherence events (\citeproc{ref-penrose1989}{Penrose,
1989}). The black hole singularity represents the exhaustion of
entropy's gradient---an endpoint where informational degrees of freedom
no longer evolve (\citeproc{ref-bekenstein1973}{Bekenstein, 1973};
\citeproc{ref-hawking1976}{Hawking, 1976}).

From this perspective, reality exists only where decoherence continues
at finite Collapse Lengths. Time, locality, and measurement all derive
from this persistence. Inside a black hole, what appears as
gravitational collapse is in fact a topological transition from
classical spacetime back into singularity space. The interior is not an
informational furnace nor a chaotic hyper-decoherent zone, but a
\textbf{coherence-starved boundary}---a terminal condition for the
physics of reality itself.

\section{7. Quantum Aliasing and Measurement
Artefacts}\label{quantum-aliasing-and-measurement-artefacts}

In the decoherence bootstrap framework, reality is not a smooth,
continuous fabric but a sequence of discrete decoherence events,
separated by the \emph{collapse length}. Just as a digital image can
misrepresent a pattern when the sampling rate is too low (aliasing), our
measurements can misinterpret physical processes when they occur near or
beyond the resolution limit set by this collapse length. This draws a
direct analogy to the Nyquist--Shannon sampling theorem, where
insufficient sampling produces artefacts not present in the underlying
signal Nyquist (\citeproc{ref-nyquist1928}{1928}).

\textbf{Aliasing in quantum systems} occurs when the temporal or spatial
resolution of decoherence is insufficient to fully track a system's
underlying state evolution. In such cases, the ``observed'' state is a
coarse-grained projection, and the interference patterns, tunneling
probabilities, and entanglement correlations that we call \emph{quantum
phenomena} may be understood as the byproducts of this undersampling.

This reinterpretation suggests that:

\begin{itemize}
\tightlist
\item
  \textbf{Wave--particle duality} emerges because measurement interacts
  with an object's decoherence sequence at irregular intervals,
  producing results that are effectively superpositions from the
  undersampled perspective.
\item
  \textbf{Quantum interference} is the observable counterpart of
  sampling a system at phase offsets relative to its internal
  decoherence rhythm, allowing multiple possible histories to overlap in
  the measurement record.
\item
  \textbf{Entanglement} represents synchronized aliasing between
  subsystems whose decoherence events remain correlated, so that
  undersampling one necessarily produces correlated undersampling in the
  other.
\item
  \textbf{Quantum tunneling} can be reframed as a sampling artefact
  where the system's state changes between decoherence events in a way
  that makes intermediate classical paths unresolvable.
\end{itemize}

From this perspective, the apparent ``nonlocality'' of quantum mechanics
does not imply that information travels instantaneously, but rather that
our observational frame rate is too low to capture all intermediate
causal links. This aligns with the decoherence bootstrap's view that
\textbf{classicality emerges only when the collapse length is short
enough} to resolve stable pointer states; in regimes where collapse
length approaches or exceeds the relevant dynamical scales, aliasing
dominates and quantum effects become manifest.\footnote{Quantum
  weirdness may be a symptom of decoherence fatigue in a universe that
  was born classical.}

Ultimately, quantum aliasing offers a unifying interpretation: the
phenomena we call ``quantum'' are not exceptions to classicality but
signatures of a deeper, discretized structure of reality, where the
limits of decoherence resolution define the limits of classical
observation Schlosshauer (\citeproc{ref-schlosshauer2007}{2007}).

\section{8. Toward a Conservation Law for
Decoherence}\label{toward-a-conservation-law-for-decoherence}

We now propose a working principle: the evolution of Collapse Length is
governed by a conservation law linking energy, entropy, and decoherence
dynamics. In this view, the universe is not only conserving energy and
momentum in the conventional sense---it is also conserving a
\textbf{joint energy--entropy quantity} that determines how
quantum-to-classical transitions proceed.

In the decoherence bootstrap framework, the path from the shortest
possible Collapse Length at the Big Bang to the infinite Collapse Length
at a black hole singularity is not arbitrary. It follows a trajectory
that optimally dissipates the initial ``heat'' of the singularity into
long-lived classical structures. The ordering of force
emergence---gravity \(\to\) strong \(\to\) electroweak \(\to\)
electromagnetic---is then interpretable as \textbf{the optimal sequence
for shedding coherence energy into increasingly specialized and low-loss
channels}.

\begin{itemize}
\tightlist
\item
  \textbf{Gravity first}: universal coupling, maximal reach, sets the
  global spacetime scaffold for further dissipation.
\item
  \textbf{Strong force}: locks matter into bound states, trapping most
  rest mass-energy efficiently and stably.
\item
  \textbf{Electroweak}: permits massive particle decay channels and
  fine-tunes charge and spin configurations.
\item
  \textbf{Electromagnetism}: final, low-energy pointer basis that
  supports persistent information encoding and radiative dissipation.
\end{itemize}

From this standpoint, what we call ``reality'' is the byproduct of a
maximal-efficiency cascade: a system that began in a state of shortest
Collapse Length---high-frequency decoherence, zero persistence---has
evolved along a path that balances energy disposal with the retention of
coherent structures.

If this conservation principle is valid, it should leave signatures in:

\begin{itemize}
\tightlist
\item
  The relative timescales of force decoupling as imprinted in CMB
  anisotropies.
\item
  The statistical distribution of bound versus unbound matter across
  cosmic history.
\item
  Deviations from \(\Lambda\)CDM predictions in early structure
  formation, reflecting constraints imposed by collapse-length
  conservation.
\end{itemize}

Formally, this could be expressed as:

\[
\frac{d}{dt}\!\left(E_{\text{classical}} + \alpha\, S_{\text{decoherence}}\right)=0
\]

where \(E_{\text{classical}}\) is the energy stored in persistent
pointer states, \(S_{\text{decoherence}}\) is the entropy associated
with collapse events, and \(\alpha\) is a scaling factor determined by
the collapse-length dynamics.

This formulation also closes the symmetry between the Big Bang and black
holes: the former is the point of maximal decoherence frequency
(Collapse Length \(\to 0\)), while the latter is the point of minimal
decoherence frequency (Collapse Length \(\to \infty\)), both
representing boundary conditions in the same conservation law.

\section{9. Related Work}\label{related-work}

Several established lines of research intersect with---and provide
context for---the present framework. Below, we summarize each strand,
note points of contact, and highlight divergences.

\begin{itemize}
\item
  \textbf{Emergent/Entropic Gravity} --- Gravity as a thermodynamic or
  entropic phenomenon has been explored by Jacobson
  (\citeproc{ref-jacobson1995}{Jacobson, 1995}), Padmanabhan
  (\citeproc{ref-padmanabhan2010}{Padmanabhan, 2010}), and Verlinde
  (\citeproc{ref-verlinde2011}{Verlinde, 2011}). These approaches treat
  spacetime curvature as an emergent effect of entropy gradients or
  information flow, often grounded in holographic principles.
  \emph{Relation}: Our framework also links gravity to fundamental
  information processes---here, the earliest and most universal
  decoherence basis. \emph{Difference}: We treat thermodynamic behavior
  as \emph{derivative} of decoherence chain sustainability, rather than
  the source of spacetime itself.
\item
  \textbf{Emergent Time from Quantum Information} --- Page \& Wootters
  (\citeproc{ref-page1983}{Page and Wootters, 1983}) and Moreva et al.
  (\citeproc{ref-moreva2014}{Moreva \emph{et al.}, 2014}) model time as
  emerging from correlations in quantum states or growth in
  computational complexity. \emph{Relation}: Our model likewise regards
  time as emergent, specifically from the sequence of stable decoherence
  events. \emph{Difference}: We tie the ``clock'' to the \emph{Collapse
  Length} between events, allowing time to decouple from coordinate
  duration and vanish in hyper-decoherence regimes such as black holes
  or the Big Bang.
\item
  \textbf{Consistent/Decoherent Histories \& Quantum Darwinism} ---
  Gell-Mann \& Hartle (\citeproc{ref-gellmann1990}{Gell-Mann and Hartle,
  1990}) and Zurek (\citeproc{ref-zurek2009}{Zurek, 2009}) explain
  classical reality as a selection of stable pointer states from quantum
  possibilities, maintained by environment-induced decoherence.
  \emph{Relation}: We adopt the pointer basis concept directly, with
  gravity as the first environment to stabilize any basis at all.
  \emph{Difference}: We frame this as a cosmological bootstrap, where
  sequential stabilization of pointer bases defines force emergence and
  accounts for dark matter.
\item
  \textbf{Gravity-Induced Collapse Models} --- Diósi
  (\citeproc{ref-diosi1987}{Diósi, 1987}) and Penrose
  (\citeproc{ref-penrose1996}{Penrose, 1996}) propose that gravity
  itself triggers wavefunction collapse, setting a fundamental scale for
  quantum/classical transition. \emph{Relation}: Our view is compatible
  with gravity being a universal decoherence channel. \emph{Difference}:
  We extend this to a full force-emergence chronology and interpret dark
  matter as matter collapsed gravitationally before EM basis formation.
\item
  \textbf{Black-Hole--Driven Decoherence} --- Danielson, Satishchandran
  \& Wald (\citeproc{ref-danielson2023}{Danielson, Satishchandran and
  Wald, 2023}, \citeproc{ref-danielson2025}{2025}), Gralla
  (\citeproc{ref-gralla2023}{Gralla, 2023}), and Biggs \& Maldacena
  (\citeproc{ref-biggs2024}{Biggs and Maldacena, 2024}) show how
  horizons and black holes decohere spatial superpositions through
  horizon quanta and field correlations. \emph{Relation}: We agree that
  black holes are extreme decoherence environments. \emph{Difference}:
  We reinterpret them as \emph{termination points} of the decoherence
  chain, where Collapse Length diverges and classicality fails to
  emerge, yielding informational turbulence rather than mere information
  loss.
\end{itemize}

\textbf{Summary of Distinction}

Our framework integrates multiple research strands into a single
ontological model centered on \textbf{Collapse Length} and the
sustainability of the decoherence chain, yielding three defining
distinctions:

\begin{itemize}
\tightlist
\item
  \textbf{Dual Extremes of Decoherence} --- Both the Big Bang and black
  hole singularities represent hyper-decoherence, yet differ
  fundamentally in origin, boundary conditions, and cosmological role.
\item
  \textbf{Opposite Causal Functions} --- The Big Bang's
  hyper-decoherence launches spacetime and the sequence of emergent
  forces; a black hole's hyper-decoherence dismantles existing coherent
  structures.
\item
  \textbf{Inverted Entropy Flow} --- The Big Bang drives entropy outward
  into the cosmos; black hole singularities draw entropy inward,
  reversing their thermodynamic roles in cosmic evolution.
\end{itemize}

\section{10. Conclusion and Forward
Directions}\label{conclusion-and-forward-directions}

The Decoherence Bootstrap Hypothesis reframes some of the most
persistent cosmological puzzles---not as problems requiring new sectors
or ad-hoc fields, but as natural consequences of decoherence dynamics
across gravitational and electromagnetic pointer bases. By treating
``collapse length'' as the fundamental measure of reality's continuity,
we arrive at a coherent framework that connects early-universe
conditions, structure formation, quantum anomalies, and the role of
black holes as terminal states of collapse.

In this view, inflation is not strictly necessary to explain the
observed isotropy and structure of the cosmos; instead, an extremely
short gravitational collapse length in the earliest moments provides the
rapid ``information reset'' and smoothing normally attributed to
inflationary expansion. Likewise, dark matter is reinterpreted as mass
decohered in the gravitational basis but not in the electromagnetic
basis, offering an observationally anchored explanation that remains
testable via astrophysical lensing and structure growth measurements.

Looking forward, this hypothesis suggests multiple directions for theory
and experiment:

\begin{enumerate}
\def\labelenumi{\arabic{enumi}.}
\tightlist
\item
  \textbf{Quantitative modeling of collapse length evolution} ---
  Develop mathematical models for how collapse length changes with
  energy density and entropy, connecting Planck-scale conditions to
  late-time cosmic structure.
\item
  \textbf{Gravitational decoherence signatures} --- Identify
  observational imprints of gravitational decoherence distinct from
  standard dark matter and inflationary models, including lensing
  profiles and CMB residuals.
\item
  \textbf{Laboratory analogs} --- Explore condensed-matter and
  quantum-optics systems where pointer basis separation can be tuned,
  enabling controlled ``collapse length'' experiments.
\item
  \textbf{Black hole interior regimes} --- Investigate whether the
  increase of collapse length toward infinity inside event horizons can
  be modeled as a time-reversed analog of the big bang's initial
  collapse.
\item
  \textbf{Integration with conservation principles} --- Test whether the
  ordering of force emergence (gravity → strong → electroweak →
  electromagnetic) can be derived from optimal pathways for dissipating
  initial energy density.
\end{enumerate}

The core strength of this approach is that it reduces reliance on
speculative entities and instead builds from first principles already
grounded in quantum mechanics and general relativity. While the
hypothesis is bold in scope, its predictions are concrete: the next step
is to formalize them into simulations and derive observational
discriminants.

If successful, the Decoherence Bootstrap Hypothesis could unify
cosmology's ``dark'' problems under a single, testable
framework---turning questions of missing matter and inflation into
questions of how reality itself comes into being.

\clearpage

\section{Methodology Appendix}\label{methodology-appendix}

\section{Appendix A --- Source Use \&
Verification}\label{appendix-a-source-use-verification}

\begin{enumerate}
\def\labelenumi{\arabic{enumi})}
\tightlist
\item
  Boltzmann (1896/1897): Responses in the Zermelo debate
\end{enumerate}

Use in manuscript --- Historical grounding for the statistical
(probabilistic) reading of the Second Law and the recurrence objection
your Boltzmann-fluctuation framing builds on. Direct vs consensus ---
Consensus anchor (established history of ideas). Where to cite --- Good
secondary overviews with excerpts and context: Steckline (1983) and
SEP/Uffink entry on Boltzmann's statistical physics.

\begin{enumerate}
\def\labelenumi{\arabic{enumi})}
\setcounter{enumi}{1}
\tightlist
\item
  Boltzmann (English access to Zermelo exchange)
\end{enumerate}

Use --- English-accessible documentation of the 1896--97 exchange to
justify wording around ``Boltzmann fluctuation''. Direct vs consensus
--- Consensus anchor. Where --- Rutgers translation notes and
bibliographic pointers to the Zermelo paper and Brush's translations.

\begin{enumerate}
\def\labelenumi{\arabic{enumi})}
\setcounter{enumi}{2}
\tightlist
\item
  Nyquist (1928): Thermal agitation in conductors
\end{enumerate}

Use --- Foundational result behind noise/aliasing analogies (Nyquist
relation) used in Sec. 7. Direct vs consensus --- Direct claim
(classical formula \& method). Where --- Phys. Rev.~32, 110 (1928) and
open PDF scans.

\begin{enumerate}
\def\labelenumi{\arabic{enumi})}
\setcounter{enumi}{3}
\tightlist
\item
  Shannon (1948/1949): Information theory \& noise
\end{enumerate}

Use --- Formal basis for sampling/communication language you use to
frame ``quantum aliasing.'' Direct vs consensus --- Consensus anchor;
when you mention continuous/noisy channels, the 1949 paper is apt. Where
--- 1948 BSTJ parts I \& II; 1949 ``Communication in the Presence of
Noise.''

\begin{enumerate}
\def\labelenumi{\arabic{enumi})}
\setcounter{enumi}{4}
\tightlist
\item
  Zwicky (1933): Redshift paper (early missing-mass insight)
\end{enumerate}

Use --- Early articulation that non-luminous matter must exceed luminous
matter; sets historical arc for ``gravitational-only'' mass. Direct vs
consensus --- Direct claim (historical primary). Where --- English
``Golden Oldies'' reprint/translation and original HPA record.

\begin{enumerate}
\def\labelenumi{\arabic{enumi})}
\setcounter{enumi}{5}
\tightlist
\item
  Zwicky (1937): On the Masses of Nebulae and of Clusters of Nebulae
\end{enumerate}

Use --- Classic Coma-cluster virial mass discrepancy; your dark-matter
section cites this as the cluster-scale anchor. Direct vs consensus ---
Direct claim. Where --- ApJ 86, 217--246; ADS full text.

\begin{enumerate}
\def\labelenumi{\arabic{enumi})}
\setcounter{enumi}{6}
\tightlist
\item
  Rubin \& Ford (1970): Andromeda rotation curve
\end{enumerate}

Use --- Galaxy-scale evidence for dark halos (flat rotation curves)
supporting your ``gravitationally decohered, non-EM'' mass. Direct vs
consensus --- Direct claim. Where --- ApJ 159, 379--403; ADS full text.

\begin{enumerate}
\def\labelenumi{\arabic{enumi})}
\setcounter{enumi}{7}
\tightlist
\item
  Penrose (1965): Singularity theorem (collapse → singularities)
\end{enumerate}

Use --- Theorem-level support for your black-hole boundary condition
(collapse length \(\to \infty\) endpoint). Direct vs consensus ---
Direct claim. Where --- PRL 14, 57--59; open PDF.

\begin{enumerate}
\def\labelenumi{\arabic{enumi})}
\setcounter{enumi}{8}
\tightlist
\item
  Gell-Mann (1964): Quark model
\end{enumerate}

Use --- Establishes the strong-sector microstructure referenced when you
sequence force emergence. Direct vs consensus --- Consensus anchor
(historical discovery paper). Where --- Physics Letters 8, 214--215; PDF
scan.

\begin{enumerate}
\def\labelenumi{\arabic{enumi})}
\setcounter{enumi}{9}
\tightlist
\item
  Higgs (1964): Gauge boson masses via SSB
\end{enumerate}

Use --- Fixes the electroweak epoch ``energy setting'' you reference
(Higgs mechanism as phase-transition marker). Direct vs consensus ---
Direct claim. Where --- PRL 13, 508--509; APS page/PDF.

\begin{enumerate}
\def\labelenumi{\arabic{enumi})}
\setcounter{enumi}{10}
\tightlist
\item
  Glashow (1961): Partial symmetries of weak interactions
\end{enumerate}

Use --- Pre-unification scaffold used in your narrative about
electroweak emergence. Direct vs consensus --- Direct claim (model
paper). Where --- Nuclear Physics 22, 579--588; ScienceDirect record /
accessible copies.

\begin{enumerate}
\def\labelenumi{\arabic{enumi})}
\setcounter{enumi}{11}
\tightlist
\item
  Weinberg (1967): A Model of Leptons
\end{enumerate}

Use --- Electroweak unification cornerstone; you cite it as the formal
consolidation preceding Higgs discovery. Direct vs consensus --- Direct
claim. Where --- PRL 19, 1264--1266; APS page / PDF copies.

\begin{enumerate}
\def\labelenumi{\arabic{enumi})}
\setcounter{enumi}{12}
\tightlist
\item
  Gross \& Wilczek (1973): Asymptotic freedom
\end{enumerate}

Use --- Strong-interaction running (high-energy weakening) backing your
sequencing of force stabilization. Direct vs consensus --- Direct claim.
Where --- PRL 30, 1343--1346; APS page/PDF.

\begin{enumerate}
\def\labelenumi{\arabic{enumi})}
\setcounter{enumi}{13}
\tightlist
\item
  Politzer (1973): Asymptotic freedom (independent)
\end{enumerate}

Use --- Independent derivation confirming \#13; strengthens the QCD
pillar in Sec. 4. Direct vs consensus --- Direct claim. Where --- PRL
30, 1346--1349; APS page/PDF.

\begin{enumerate}
\def\labelenumi{\arabic{enumi})}
\setcounter{enumi}{14}
\tightlist
\item
  Weinberg (1974): Symmetry restoration at high temperature
\end{enumerate}

Use --- Thermal field-theory basis for your ``decoherence
thresholds''/epoch sequencing (high-T restoration). Direct vs consensus
--- Direct claim. Where --- Phys. Rev.~D 9, 3357--3378; APS/OSTI copies.

\begin{enumerate}
\def\labelenumi{\arabic{enumi})}
\setcounter{enumi}{15}
\tightlist
\item
  Hawking \& Ellis (1973): The Large Scale Structure of Space-Time
\end{enumerate}

Use --- Global GR framework \& singularity/collapse chapters that your
black-hole boundary condition leans on. Direct vs consensus ---
Consensus anchor (graduate-level treatise). Where --- Cambridge
monograph; see Ch. 8 ``Space-time singularities,'' Ch. 9 ``Gravitational
collapse and black holes.''

\begin{center}\rule{0.5\linewidth}{0.5pt}\end{center}

\emph{(Next: entries from Misner, Thorne \& Wheeler (1973) onward---)}

\subsection{Appendix B --- Courses of
Study}\label{appendix-b-courses-of-study}

\subsubsection{Legend}\label{legend}

\begin{longtable}[]{@{}ll@{}}
\toprule\noalign{}
Code & Topic Area \\
\midrule\noalign{}
\endhead
\bottomrule\noalign{}
\endlastfoot
01 & Thermodynamics \\
02 & Information Theory \\
03 & Dark Matter Foundations \\
04 & General Relativity Foundations \\
05 & Quantum Field Theory Foundations \\
06 & High-Temp Symmetry Physics \\
07 & Black Hole Thermodynamics \& Physics \\
08 & Cosmology \& Inflation \\
09 & Quantum Foundations \& Time \\
10 & Complexity \& Self-Organization \\
11 & Quantum Gravity \& Decoherence \\
12 & Thermodynamics of Spacetime \& Emergent Gravity \\
13 & Quantum Information \& Computation \\
14 & Decoherence Theory \\
15 & Time, Entropy \& Cosmology \\
\end{longtable}

\subsubsection{Mastery Tracker}\label{mastery-tracker}

\begin{longtable}[]{@{}
  >{\raggedright\arraybackslash}p{(\linewidth - 8\tabcolsep) * \real{0.1667}}
  >{\raggedright\arraybackslash}p{(\linewidth - 8\tabcolsep) * \real{0.4000}}
  >{\raggedright\arraybackslash}p{(\linewidth - 8\tabcolsep) * \real{0.1333}}
  >{\raggedright\arraybackslash}p{(\linewidth - 8\tabcolsep) * \real{0.1333}}
  >{\raggedright\arraybackslash}p{(\linewidth - 8\tabcolsep) * \real{0.1667}}@{}}
\toprule\noalign{}
\begin{minipage}[b]{\linewidth}\raggedright
Order
\end{minipage} & \begin{minipage}[b]{\linewidth}\raggedright
Citation Key
\end{minipage} & \begin{minipage}[b]{\linewidth}\raggedright
Year
\end{minipage} & \begin{minipage}[b]{\linewidth}\raggedright
Code
\end{minipage} & \begin{minipage}[b]{\linewidth}\raggedright
Notes
\end{minipage} \\
\midrule\noalign{}
\endhead
\bottomrule\noalign{}
\endlastfoot
1 & boltzmann1896 & 1896 & 01 & Entropy arguments, statistical mechanics
foundations. \\
2 & boltzmann1896english & 1966 & 01 & English translation of
Boltzmann's reply to Zermelo. \\
3 & nyquist1928 & 1928 & 02 & Telegraph transmission theory; precursors
to Shannon. \\
4 & shannon1949 & 1949 & 02 & Communication in presence of noise;
information entropy. \\
5 & zwicky1933 & 1933 & 03 & Galaxy cluster redshift anomalies. \\
6 & zwicky1937 & 1937 & 03 & Masses of nebulae; further DM evidence. \\
7 & rubin1970 & 1970 & 03 & Andromeda rotation curves. \\
8 & penrose1965 & 1965 & 04 & Singularity theorems, cosmic
censorship. \\
9 & gellmann1964 & 1964 & 05 & Eightfold way; symmetry
classification. \\
10 & higgs1964 & 1964 & 05 & Mass generation for gauge bosons. \\
11 & glashow1961 & 1961 & 05 & Partial symmetries of weak
interactions. \\
12 & weinberg1967 & 1967 & 05 & Electroweak unification. \\
13 & gross1973 & 1973 & 05 & Asymptotic freedom in non-Abelian gauge
theories. \\
14 & politzer1973 & 1973 & 05 & Asymptotic freedom results. \\
15 & weinberg1974 & 1974 & 06 & Gauge/global symmetries at high
temperature. \\
16 & hawking1973 & 1973 & 04 & Large scale structure of spacetime. \\
17 & misner1973 & 1973 & 04 & Gravitation (MTW). \\
18 & bekenstein1973 & 1973 & 07 & Black hole entropy formula. \\
19 & hawking1976 & 1976 & 07 & Breakdown of predictability; information
paradox. \\
20 & unruh1976 & 1976 & 07 & Unruh effect derivation. \\
21 & guth1981 & 1981 & 08 & Inflationary universe proposal. \\
22 & mukhanov1981 & 1981 & 08 & Quantum fluctuations in early
universe. \\
23 & linde1982 & 1982 & 08 & New inflationary universe scenario. \\
24 & page1983 & 1983 & 09 & Evolution without evolution
(Page--Wootters). \\
25 & blumenthal1984 & 1984 & 03 & Cold dark matter model. \\
26 & prigogine1984 & 1984 & 10 & Order out of chaos;
self-organization. \\
27 & diosi1987 & 1987 & 11 & Gravitationally induced decoherence
model. \\
28 & penrose1989 & 1989 & 11 & The Emperor's New Mind; OR model
beginnings. \\
29 & gellmann-hartle1990 & 1990 & 09 & Consistent histories
formulation. \\
30 & gellmann1990 & 1990 & 09 & Duplicate citation for consistent
histories. \\
31 & penrose1994 & 1994 & 11 & Shadows of the Mind; OR model
expanded. \\
32 & penrose1996 & 1996 & 11 & Gravity's role in quantum state
reduction. \\
33 & jacobson1995 & 1995 & 12 & Einstein equation as equation of
state. \\
34 & lloyd2000 & 2000 & 13 & Limits to computation. \\
35 & ng2003 & 2003 & 11 & Spacetime foam, holography, nonlocality. \\
36 & zurek2003 & 2003 & 14 & Einselection and quantum-classical
transition. \\
37 & wilczek2005 & 2005 & 05 & Asymptotic freedom retrospective. \\
38 & clowe2006 & 2006 & 03 & Bullet Cluster observations. \\
39 & schlosshauer2007 & 2007 & 14 & Comprehensive review of decoherence
theory. \\
40 & carroll2008 & 2010 & 15 & From Eternity to Here. \\
41 & padmanabhan2010 & 2010 & 12 & New insights into gravity as
thermodynamic. \\
42 & verlinde2011 & 2011 & 12 & Gravity as entropic force. \\
43 & brandenberger2011 & 2011 & 08 & Alternatives to inflationary
paradigm. \\
44 & moreva2014 & 2014 & 09 & Time from quantum entanglement
experiment. \\
45 & peskin2015 & 2015 & 05 & Electroweak symmetry breaking concepts. \\
46 & susskind2018 & 2018 & 07 & Three lectures on complexity and BHs. \\
47 & bertone2018 & 2018 & 03 & History of dark matter research. \\
48 & gralla2023 & 2023 & 11 & Kerr horizon decoherence. \\
49 & danielson2023 & 2023 & 11 & Killing horizons decohere
superpositions. \\
50 & biggs2024 & 2024 & 11 & Comparing BH vs ordinary matter
decoherence. \\
51 & danielson2025 & 2025 & 11 & Local decoherence by BHs and bodies. \\
\end{longtable}

\newpage

\protect\phantomsection\label{refs}
\begin{CSLReferences}{0}{1}
\bibitem[\citeproctext]{ref-baumann2009}
Baumann, D. (2009) \emph{TASI lectures on inflation}. World Scientific
(TASI 2009 lecture notes). Available at:
\url{https://arxiv.org/abs/0907.5424}.

\bibitem[\citeproctext]{ref-bekenstein1973}
Bekenstein, J.D. (1973) {`Black holes and entropy'}, \emph{Physical
Review D}, 7, pp. 2333--2346. Available at:
\url{https://doi.org/10.1103/PhysRevD.7.2333}.

\bibitem[\citeproctext]{ref-bertone2018}
Bertone, G. and Hooper, D. (2018) {`History of dark matter'},
\emph{Reviews of Modern Physics}, 90(4), p. 045002. Available at:
\url{https://doi.org/10.1103/RevModPhys.90.045002}.

\bibitem[\citeproctext]{ref-biggs2024}
Biggs, A. and Maldacena, J. (2024) {`Comparing the decoherence effects
due to black holes versus ordinary matter'}. Available at:
\url{https://arxiv.org/abs/2405.02227}.

\bibitem[\citeproctext]{ref-blumenthal1984}
Blumenthal, G.R. \emph{et al.} (1984) {`Formation of galaxies and
large-scale structure with cold dark matter'}, \emph{Nature}, 311, pp.
517--525. Available at: \url{https://doi.org/10.1038/311517a0}.

\bibitem[\citeproctext]{ref-boltzmann1896}
Boltzmann, L. (1896) {`Entgegnung auf die w{ä}rmetheoretischen
betrachtungen des hrn. E. zermelo'}, \emph{Annalen der Physik}, 293(3),
pp. 773--784. Available at:
\url{https://doi.org/10.1002/andp.18962930316}.

\bibitem[\citeproctext]{ref-boltzmann1896english}
Boltzmann, L. (1966) {`Reply to zermelo's remarks on the theory of
heat'}, in S.G. Brush (ed.) \emph{The kinetic theory of gases: An
anthology of classic papers with historical commentary}. Pergamon Press,
pp. 262--270.

\bibitem[\citeproctext]{ref-brandenberger2011}
Brandenberger, R.H. (2011) {`Alternatives to the inflationary paradigm
of structure formation'}, \emph{International Journal of Modern Physics:
Conference Series}, 01, pp. 67--79. Available at:
\url{https://doi.org/10.1142/S2010194511000109}.

\bibitem[\citeproctext]{ref-carroll2008}
Carroll, S. (2010) \emph{From eternity to here: The quest for the
ultimate theory of time}. New York: Dutton.

\bibitem[\citeproctext]{ref-clowe2006}
Clowe, D. \emph{et al.} (2006) {`A direct empirical proof of the
existence of dark matter'}, \emph{The Astrophysical Journal Letters},
648(2), pp. L109--L113. Available at:
\url{https://doi.org/10.1086/508162}.

\bibitem[\citeproctext]{ref-danielson2023}
Danielson, D.L., Satishchandran, G. and Wald, R.M. (2023) {`Killing
horizons decohere quantum superpositions'}, \emph{Physical Review D},
108(2), p. 025007. Available at:
\url{https://doi.org/10.1103/PhysRevD.108.025007}.

\bibitem[\citeproctext]{ref-danielson2025}
Danielson, D.L., Satishchandran, G. and Wald, R.M. (2025) {`Local
description of decoherence of quantum superpositions by black holes and
other bodies'}, \emph{Physical Review D}, 111(2), p. 025014. Available
at: \url{https://doi.org/10.1103/PhysRevD.111.025014}.

\bibitem[\citeproctext]{ref-diosi1987}
Diósi, L. (1987) {`A universal master equation for the gravitational
violation of quantum mechanics'}, \emph{Physics Letters A}, 120(8), pp.
377--381. Available at:
\url{https://doi.org/10.1016/0375-9601(87)90681-5}.

\bibitem[\citeproctext]{ref-gellmann1990}
Gell-Mann, M. and Hartle, J.B. (1990) {`Quantum mechanics in the light
of quantum cosmology'}, in W.H. Zurek (ed.) \emph{Complexity, entropy
and the physics of information}. Addison-Wesley, pp. 425--458.

\bibitem[\citeproctext]{ref-glashow1961}
Glashow, S.L. (1961) {`Partial-symmetries of weak interactions'},
\emph{Nuclear Physics}, 22(4), pp. 579--588. Available at:
\url{https://doi.org/10.1016/0029-5582(61)90469-2}.

\bibitem[\citeproctext]{ref-gralla2023}
Gralla, S.E. (2023) {`Decoherence from horizons: General considerations
and applications to {Kerr}'}. Available at:
\url{https://arxiv.org/abs/2311.11461}.

\bibitem[\citeproctext]{ref-gross1973}
Gross, D.J. and Wilczek, F. (1973) {`Ultraviolet behavior of non-abelian
gauge theories'}, \emph{Physical Review Letters}, 30(26), pp.
1343--1346. Available at:
\url{https://doi.org/10.1103/PhysRevLett.30.1343}.

\bibitem[\citeproctext]{ref-guth1981}
Guth, A.H. (1981) {`Inflationary universe: A possible solution to the
horizon and flatness problems'}, \emph{Physical Review D}, 23(2), pp.
347--356. Available at: \url{https://doi.org/10.1103/PhysRevD.23.347}.

\bibitem[\citeproctext]{ref-hawking1976}
Hawking, S.W. (1976) {`Breakdown of predictability in gravitational
collapse'}, \emph{Physical Review D}, 14, pp. 2460--2473. Available at:
\url{https://doi.org/10.1103/PhysRevD.14.2460}.

\bibitem[\citeproctext]{ref-hawking1973}
Hawking, S.W. and Ellis, G.F.R. (1973) \emph{The large scale structure
of space-time}. Cambridge University Press. Available at:
\url{https://doi.org/10.1017/CBO9780511524646}.

\bibitem[\citeproctext]{ref-higgs1964}
Higgs, P.W. (1964) {`Broken symmetries and the masses of gauge bosons'},
\emph{Physical Review Letters}, 13(16), pp. 508--509. Available at:
\url{https://doi.org/10.1103/PhysRevLett.13.508}.

\bibitem[\citeproctext]{ref-jacobson1995}
Jacobson, T. (1995) {`Thermodynamics of spacetime: The {Einstein}
equation of state'}, \emph{Physical Review Letters}, 75, pp. 1260--1263.
Available at: \url{https://doi.org/10.1103/PhysRevLett.75.1260}.

\bibitem[\citeproctext]{ref-linde1982}
Linde, A.D. (1982) {`A new inflationary universe scenario: A possible
solution of the horizon, flatness, homogeneity, isotropy and primordial
monopole problems'}, \emph{Physics Letters B}, 108(6), pp. 389--393.
Available at: \url{https://doi.org/10.1016/0370-2693(82)91219-9}.

\bibitem[\citeproctext]{ref-misner1973}
Misner, C.W., Thorne, K.S. and Wheeler, J.A. (1973) \emph{Gravitation}.
W. H. Freeman.

\bibitem[\citeproctext]{ref-moreva2014}
Moreva, E. \emph{et al.} (2014) {`Time from quantum entanglement: An
experimental illustration'}, \emph{Physical Review A}, 89(5), p. 052122.
Available at: \url{https://doi.org/10.1103/PhysRevA.89.052122}.

\bibitem[\citeproctext]{ref-mukhanov1981}
Mukhanov, V.F. and Chibisov, G.V. (1981) {`Quantum fluctuations and a
nonsingular universe'}, \emph{JETP Letters}, 33, pp. 532--535. Available
at: \url{http://www.jetpletters.ac.ru/ps/1532/article_23403.shtml}.

\bibitem[\citeproctext]{ref-nyquist1928}
Nyquist, H. (1928) {`Certain topics in telegraph transmission theory'},
\emph{Transactions of the American Institute of Electrical Engineers},
47(2), pp. 617--644. Available at:
\url{https://doi.org/10.1109/T-AIEE.1928.5055024}.

\bibitem[\citeproctext]{ref-padmanabhan2010}
Padmanabhan, T. (2010) {`Thermodynamical aspects of gravity: New
insights'}, \emph{Reports on Progress in Physics}, 73(4), p. 046901.
Available at: \url{https://doi.org/10.1088/0034-4885/73/4/046901}.

\bibitem[\citeproctext]{ref-page1983}
Page, D.N. and Wootters, W.K. (1983) {`Evolution without evolution:
Dynamics described by stationary observables'}, \emph{Physical Review
D}, 27(12), pp. 2885--2892. Available at:
\url{https://doi.org/10.1103/PhysRevD.27.2885}.

\bibitem[\citeproctext]{ref-penrose1965}
Penrose, R. (1965) {`Gravitational collapse and space-time
singularities'}, \emph{Physical Review Letters}, 14, pp. 57--59.
Available at: \url{https://doi.org/10.1103/PhysRevLett.14.57}.

\bibitem[\citeproctext]{ref-penrose1989}
Penrose, R. (1989) \emph{The emperor's new mind}. Oxford University
Press.

\bibitem[\citeproctext]{ref-penrose1994}
Penrose, R. (1994) \emph{Shadows of the mind: A search for the missing
science of consciousness}. Oxford, UK: Oxford University Press.

\bibitem[\citeproctext]{ref-penrose1996}
Penrose, R. (1996) {`On gravity's role in quantum state reduction'},
\emph{General Relativity and Gravitation}, 28, pp. 581--600. Available
at: \url{https://doi.org/10.1007/BF02105068}.

\bibitem[\citeproctext]{ref-politzer1973}
Politzer, H.D. (1973) {`Reliable perturbative results for strong
interactions?'}, \emph{Physical Review Letters}, 30(26), pp. 1346--1349.
Available at: \url{https://doi.org/10.1103/PhysRevLett.30.1346}.

\bibitem[\citeproctext]{ref-prigogine1984}
Prigogine, I. and Stengers, I. (1984) \emph{Order out of chaos: Man's
new dialogue with nature}. Bantam Books.

\bibitem[\citeproctext]{ref-rubin1970}
Rubin, V.C. and Ford Jr, W.K. (1970) {`Rotation of the andromeda nebula
from a spectroscopic survey of emission regions'}, \emph{The
Astrophysical Journal}, 159, pp. 379--403. Available at:
\url{https://doi.org/10.1086/150317}.

\bibitem[\citeproctext]{ref-schlosshauer2007}
Schlosshauer, M. (2007) \emph{Decoherence and the quantum-to-classical
transition}. Berlin: Springer. Available at:
\url{https://doi.org/10.1007/978-3-540-35775-9}.

\bibitem[\citeproctext]{ref-shannon1949}
Shannon, C.E. (1949) {`Communication in the presence of noise'},
\emph{Proceedings of the IRE}, 37(1), pp. 10--21. Available at:
\url{https://doi.org/10.1109/JRPROC.1949.232969}.

\bibitem[\citeproctext]{ref-steinhardt2002}
Steinhardt, P.J. and Turok, N. (2002) {`A cyclic model of the
universe'}, \emph{Science}, 296(5572), pp. 1436--1439. Available at:
\url{https://doi.org/10.1126/science.1070462}.

\bibitem[\citeproctext]{ref-rubin1980}
Vera C. Rubin, N. Thonnard and Jr., W.K.F. (1980) {`Rotational
properties of 21 SC galaxies with a large range of luminosities and
radii, from NGC 4605 /r = 4kpc/ to UGC 2885 /r = 122kpc/'},
\emph{Astrophysical Journal}, 238, pp. 471--487.

\bibitem[\citeproctext]{ref-verlinde2011}
Verlinde, E. (2011) {`On the origin of gravity and the laws of
{Newton}'}, \emph{Journal of High Energy Physics}, 2011(4), p. 29.
Available at: \url{https://doi.org/10.1007/JHEP04(2011)029}.

\bibitem[\citeproctext]{ref-weinberg1967}
Weinberg, S. (1967) {`A model of leptons'}, \emph{Physical Review
Letters}, 19(21), pp. 1264--1266. Available at:
\url{https://doi.org/10.1103/PhysRevLett.19.1264}.

\bibitem[\citeproctext]{ref-weinberg1974}
Weinberg, S. (1974) {`Gauge and global symmetries at high temperature'},
\emph{Physical Review D}, 9(12), pp. 3357--3378. Available at:
\url{https://doi.org/10.1103/PhysRevD.9.3357}.

\bibitem[\citeproctext]{ref-zurek2003}
Zurek, W.H. (2003) {`Decoherence, einselection, and the quantum origins
of the classical'}, \emph{Reviews of Modern Physics}, 75(3), pp.
715--775. Available at: \url{https://doi.org/10.1103/RevModPhys.75.715}.

\bibitem[\citeproctext]{ref-zurek2009}
Zurek, W.H. (2009) {`{Quantum Darwinism}'}, \emph{Nature Physics}, 5,
pp. 181--188. Available at: \url{https://doi.org/10.1038/nphys1202}.

\bibitem[\citeproctext]{ref-zwicky1933}
Zwicky, F. (1933) {`Die rotverschiebung von extragalaktischen nebeln'},
\emph{Helvetica Physica Acta}, 6, pp. 110--127.

\bibitem[\citeproctext]{ref-zwicky1937}
Zwicky, F. (1937) {`On the masses of nebulae and of clusters of
nebulae'}, \emph{The Astrophysical Journal}, 86, pp. 217--246. Available
at: \url{https://doi.org/10.1086/143864}.

\end{CSLReferences}

\end{document}
